\documentclass[a4paper,12pt]{article}

\usepackage[UTF8]{ctex}
\usepackage{amsmath}
\usepackage{bbding}
\usepackage{booktabs}
\usepackage{caption}
\usepackage[font={small}]{caption}
\usepackage{diagbox}
\usepackage{enumerate}
\usepackage{enumitem}
\usepackage{fancyhdr}
\usepackage{float}
\usepackage{geometry}
\usepackage{graphicx}
\usepackage{multicol}
\usepackage{subfigure}
\usepackage{tabularx}
\usepackage{titlesec}
\usepackage{listings}
\usepackage{fontspec}
\usepackage{xcolor}

%%%%%%%%%%%%%%%%%%%%%%%%%%%%%%%%%%%%%%%%
% 实验信息
%%%%%%%%%%%%%%%%%%%%%%%%%%%%%%%%%%%%%%%%
\newcommand{\theTitle}{实验名称}
\newcommand{\theAuthorName}{姓名}
\newcommand{\theAuthorID}{学号}
\newcommand{\theType}{培养类型}
\newcommand{\theGrade}{年级}
\newcommand{\theMajor}{专业}
\newcommand{\theDepartment}{所属学院}
\newcommand{\theTutor}{指导教员}
\newcommand{\theTitleOfTutor}{职称}
\newcommand{\theLab}{实验室}
\newcommand{\theDate}{实验日期}

%%%%%%%%%%%%%%%%%%%%%%%%%%%%%%%%%%%%%%%%
% 排版
%%%%%%%%%%%%%%%%%%%%%%%%%%%%%%%%%%%%%%%%

% A4 页边距 
\geometry{a4paper,left=3cm,right=3cm,top=3cm,bottom=3cm}

% 字体
\setCJKmainfont{SimSun}
\setmainfont{Times New Roman}
\setCJKsansfont{SimHei}
\setCJKmonofont{SimSun}
\punctstyle{kaiming}
\setmonofont{Consolas}

% 代码块样式
\lstset{
 columns=fixed,       
 numbers=left,                                        % 在左侧显示行号
 numberstyle=\tiny\color{gray},                       % 设定行号格式
 frame=single,                                        % 设定背景边框
 backgroundcolor=\color[RGB]{245,245,244},            % 设定背景颜色
 keywordstyle=\color[RGB]{40,40,255},                 % 设定关键字颜色
 numberstyle=\footnotesize\color{darkgray},           
 commentstyle=\color{red!50!green!50!blue!50},        % 设置代码注释的格式
%  stringstyle=\rmfamily\slshape\color[RGB]{128,0,0},   % 设置字符串格式
 showstringspaces=false,                              % 不显示字符串中的空格
 language=c++,                                        % 设置语言
 basicstyle=\ttfamily,
 breaklines=true
}

% 行间距
\renewcommand{\baselinestretch}{1.0}

% 页码
\pagestyle{fancy}
\fancyhf{}
\renewcommand{\headrulewidth}{0pt}
\setlength{\footskip}{2.5cm}
\cfoot{\zihao{-5}{\thepage}}

% 四级标题
\titleformat{\section}[block]{\heiti\zihao{4}}{\chinese{section}、}{1em}{}[]
\titleformat{\subsection}[block]{\heiti\zihao{-4}}{(\chinese{subsection})}{1em}{}[]
\titleformat{\subsubsection}[block]{\heiti\zihao{-4}}{\arabic{subsubsection}.}{1em}{}[]
\titleformat{\paragraph}[block]{\songti\zihao{-4}}{(\arabic{section})}{1em}{}[]

% 有序列表
\setlist[enumerate,1]{label=\arabic*.}
\setlist[enumerate,2]{label=(\arabic*)}

% 正文插图、表格中的文字字号均为5号。
\captionsetup{font={small}}

\title{本科实验报告}

\begin{document}

%%%%%%%%%%%%%%%%%%%%%%%%%%%%%%%%%%%%%%%%
% 封面
%%%%%%%%%%%%%%%%%%%%%%%%%%%%%%%%%%%%%%%%
\begin{titlepage}
\vspace*{3\baselineskip} 

\begin{center}
    \songti\fontsize{30.0pt}{50.0pt}
    \textbf{本\quad 科\quad 实\quad 验\quad 报\quad 告}
\end{center}

\vspace*{2\baselineskip} 

\begin{table}[H]
\centering
\zihao{3}
\begin{tabular}{ccc}
    实验名称 & : & \theTitle \\ \cline{3-3}
\end{tabular}
\end{table}

\vspace*{3\baselineskip} 

\begin{table}[H]
\renewcommand\arraystretch{1.5} 
\centering
\zihao{4}
\begin{tabular}{cccccc}
学\qquad 员 & : & \theAuthorName & 学\qquad 号 & : & \theAuthorID \\ \cline{3-3} \cline{6-6} 
培养类型 & : & \theType & 年\qquad 级 & : & \theGrade \\  \cline{3-3} \cline{6-6} 
专\qquad 业 & : & \theMajor & 所属学院 & : & \theDepartment \\  \cline{3-3} \cline{6-6} 
指导教员 & : & \theTutor & 职\qquad 称 & : & \theTitleOfTutor \\  \cline{3-3} \cline{6-6} 
实\; 验\; 室 & : & \theLab & 实验日期 & : & \theDate \\ \cline{3-3} \cline{6-6} 
\end{tabular}
\end{table}

\vspace*{10\baselineskip} 
\begin{center}
    \heiti\zihao{-3}
    国防科学技术大学训练部制
\end{center}
\end{titlepage}

%%%%%%%%%%%%%%%%%%%%%%%%%%%%%%%%%%%%%%%%
% 填写说明
%%%%%%%%%%%%%%%%%%%%%%%%%%%%%%%%%%%%%%%%
\begin{center}
    \heiti\zihao{4}
    《本科实验报告》填写说明
\end{center}

\vspace*{1\baselineskip} 

\begin{enumerate}
\songti\zihao{-4}
\item 学员完成人才培养方案和课程标准要所要求的每个实验后,均须提交实验报告。
\item 实验报告封面必须打印,报告内容可以手写或打印。
\item 实验报告内容编排及打印应符合以下要求:
    \begin{enumerate}
    \fangsong\zihao{-4}
    \item 采用A4(21cm×29.7cm)白色复印纸,单面黑字打印。上下左右各侧的页边距均为3cm;缺省文档网格:字号为小4号,中文为宋体,英文和阿拉伯数字为Times New Roman,每页30行,每行36字;页脚距边界为2.5cm,页码置于页脚、居中,采用小5号阿拉伯数字从1开始连续编排,封面不编页码。
    \item 报告正文最多可设四级标题,字体均为黑体,第一级标题字号为4号,其余各级标题为小4号;标题序号第一级用“一、”、“二、”……,第二级用“(一)”、“(二)” ……,第三级用“1.”、“2.” ……,第四级用“(1)”、“(2)” ……,分别按序连续编排。
    \item 正文插图、表格中的文字字号均为5号。
    \end{enumerate}
\end{enumerate}

\newpage

%%%%%%%%%%%%%%%%%%%%%%%%%%%%%%%%%%%%%%%%
% 正文
%%%%%%%%%%%%%%%%%%%%%%%%%%%%%%%%%%%%%%%%
\section{实验目的和要求}


\section{实验内容和原理}


\section{操作方法与实验步骤}
<写出实验操作的总体思路、操作规范和主要注意事项;按顺序记录实验中每一个环节和实验现象。画出必要的实验装置结构示意图,并配以相应文字说明>


\section{实验结果与分析(可选)}
<说明分析方法(逻辑分析、系统科学分析、模糊数学分析或统计分析的方法等),对原始数据进行分析和处理,写出明确的实验结果,并说明其可靠程度;>



\section{问题与建议(可选)}
<对实验过程中出现的问题进行描述、分析,提出解决思路和方法,无法解决的,要说明原因;记录实验心得体会,提出建议。>



\end{document}
